%%%%%%%%%%%%%%%%%%%%%%%%%%%%%%%%%%%%%%%%%%%%%%%%%%%%%%%%%%%%%%%%%%%%%%%%%%%%%%%%
%2345678901234567890123456789012345678901234567890123456789012345678901234567890
%        1         2         3         4         5         6         7         8

\documentclass[letterpaper, 10 pt, conference]{ieeeconf}  % Comment this line out if you need a4paper

%\documentclass[a4paper, 10pt, conference]{ieeeconf}      % Use this line for a4 paper

\IEEEoverridecommandlockouts                              % This command is only needed if 
                                                          % you want to use the \thanks command

\overrideIEEEmargins                                      % Needed to meet printer requirements.

% See the \addtolength command later in the file to balance the column lengths
% on the last page of the document

% The following packages can be found on http:\\www.ctan.org
%\usepackage{graphics} % for pdf, bitmapped graphics files
%\usepackage{epsfig} % for postscript graphics files
%\usepackage{mathptmx} % assumes new font selection scheme installed
%\usepackage{times} % assumes new font selection scheme installed
%\usepackage{amsmath} % assumes amsmath package installed
%\usepackage{amssymb}  % assumes amsmath package installed

\title{\LARGE \bf
Trajectory planning using Dijkstra search algorithm and moving
least square smooth interpolator* }


\author{Renan S. Freitas$^{1}$, Eduardo E. M. Soares$^{2}$ and Ramon R.
Costa$^{3}$% <-this
% % stops a space
\thanks{*This work was supported by Energia Sustent{\'a}vel do Brasil and the ANEEL R\&D program (contract
COPPETEC/UFRJ JIRAU 09/15 6631-0003/2015)}% <-this % stops a space
\thanks{$^{1}$Renan S. Freitas is with 13Robotics Robotica Ltda., and
with the Department of Electrical Engineering, Federal University of Rio de
Janeiro, Brazil {\tt\small renan.freitas@13robotics.com}}%
\thanks{$^{2}$Eduardo E. M. Soares is with 13Robotics Robotica Ltda, Brazil}%
\thanks{$^{2}$Ramon R. Costa is with Federal University of Rio de
Janeiro, Brazil}}


\begin{document}



\maketitle
\thispagestyle{empty}
\pagestyle{empty}


% %%%%%%%%%%%%%%%%%%%%%%%%%%%%%%%%%%%%%%%%%%%%%%%%%%%%%%%%%%%%%%%%%%%%%%%%%%%%%%%
\begin{abstract}
There are several industrial applications for robotic manipulators that require
precomputed trajectory generation and simulation, such as high precision
welding, machining, painting and thermal spraying. On those applications,
millimeter-spaced waypoints with precise end-effector orientations are generated
on 3-D freeform surfaces, they are processed using inverse kinematics, and,
traditionally, random sampling-based methods or local/global optimization
algorithms compute the collision-free path in the joint-space. In this paper, we
explore and propose solutions for the four main problems involved in trajectory
planning: waypoints generation in cartesian coordinates on 3-D freeform surfaces
using radial basis function (RBF); computation of inverse kinematics for
non-zero tolerance on end-effector position/orientation; joint-space trajectory
generation using Dijkstra's algorithm to minimize velocities; and the jerk
reduction using the moving least square (MLS) smooth interpolator. The case
study for the developed method is the EMMA project$^*$, a robotic system for
\textit{in situ} high velocity oxy-fuel coating (HVOF) of hydraulic turbine
blades. Simulations and experimental results with a real manipulator highlight
the efficacy of the methodology and the impact of the solution for similar
problems.
\end{abstract}


%%%%%%%%%%%%%%%%%%%%%%%%%%%%%%%%%%%%%%%%%%%%%%%%%%%%%%%%%%%%%%%%%%%%%%%%%%%%%%%%
\section{INTRODUCTION}\label{sec:intro}

Robotic arms are used in several industrial applications, such as pick-and-place
and assembly, and some high precision tasks as welding, surgical assistant, and
thermal spraying (HVOF). The EMMA project \cite{c1}, case study and motivation
for this paper, is a robotic system to perform HVOF on hydraulic turbine blades
within the turbine environment. The HVOF technique reduces the mechanical wear,
thus increasing blade's life cycle, and turbine efficiency. The process
comprises a gun from which metallic particles are sprayed through a 900$^o$C
degrees torch such that they are partially melted and then deposited on the
surface of the blade by impact. For a regular coating layer, the task specifies
$3\,mm$ spacing between trajectories, the spray gun must remain normal
($90^\circ \pm 30^\circ$) to the surface, keep a $220\,mm \pm 20\,mm $ distance
from it, and operate at a constant $40\,m/min$ speed.

Since the HVOF task specifies millimeter-spaced positions and accurate
end-effector orientations, the robot must be off-line programmed, that is,
waypoints must be precomputed and simulated. The task space trajectory
generation is widely studied, as presented in \cite{c2}, a survey of papers for
techniques of freeform surfaces reconstruction. \cite{c3} and \cite{c4}
proposed different strategies with polynomial parametrization for path planning
on painting applications. However, the related works are not concerned
about high precision or they are below the HVOF requirements. In
section~\ref{sec:rbf} we present the proposed solution for high precision
operational space trajectory generation for a robotic arm on freeform surfaces
with complicated shape or topology in 3-D space, using the implicit representation of radial basis
function (RBF) \cite{c5} for surface modelling, and the marching method
algorithm \cite{c6}.

Generally, task space waypoints (TS-waypoints) for robotic arms are sequences of
10-sized tuples composed by 3-D coordinates $(x,y,z)$, normal vectors
$(n_x,n_y,n_z)$ (orientation), velocities $(v_x,v_y,v_z)$ and delta times
$(dt)$. TS-waypoints are translated into joint-space waypoints (JS-waypoints) by
inverse kinematics (IK), then a trajectory planning technique (planner) is used
to optimize the path. Random sampling-based methods, as the Rapidly-exploring
random tree (RRT), are widely used planners to compute the collision-free path
that connects two distinct JS-waypoints. \cite{c7} and \cite{c8} study the case
where the end-effector path is not predetermined. We consider the problem of
computing the JS-waypoints given a predetermined end-effector path
(TS-waypoints), which depends on the manipulator's degree of freedom (DOF) and on the task
constraints. In the specific case of the HVOF task, a 6-DOF manipulator is used,
thus if the TS-waypoint is in the arm's workspace, there is at least one
JS-waypoint solution. However, the orientation and position tolerance produce
infinite IK solutions for each TS-waypoint. \cite{c9} solves the problem of
finding configurations for a redundant mobile base robot, similar to the
proposed solution presented in section~\ref{sec:dijkstra}, which generates random
configurations and developes a greedy planner (Dijkstra).

A dense sequence of TS-waypoints would produce substantial random configurations
(JS-waypoints), thus extensive and probably computationally infeasible solutions
for a greedy planner. On the other hand, sparse TS-waypoints may not guarantee
the task precision requirement or it may generate acceleration jumps (jerky
trajectories). In section~\ref{sec:mls}, we introduce the moving least square
(MLS) smooth interpolator as a solution to the jerk minimization problem, taking
advantage of the HVOF process tolerances (position and orientation). \cite{c10}
presents a data-driven approach and uses B-spline interpolation-based method to
minimize jerk. \cite{c11} considers a shortcutting heuristic to smooth jerky
trajectories, using parabolic and straight-line curves. The MLS approach has
several advantages over cubic splines and B-splines, as it is $C^\infty$
(differentiable for all degrees of differentiation), it has intuitive and easy
tunning parameters, and it is an extremely powerful technique for function and
surface fitting \cite{c11}.

% MUDAR
The general software solution is a platform where the operator can test
robot's trajectory planning, perform environment simulations with collision
avoidance, manipulator's torques and velocities analysis, and test robot control
techniques. We integrate the developed framework to the Open
Robotics Automation Virtual Environment (OpenRave) \cite{c12} for a general
solution of the EMMA software, and similar problems. In
section~\ref{sec:results}, overall simulations and experimental results show
that the developed framework is an efficient high precision path planning.
% NUMBERS here

\section{OPERATIONAL SPACE HIGH PRECISION TRAJECTORY GENERATION}\label{sec:rbf}
The Operational Space Trajectory Generation (OSTG) is the first component of the
developed framework. It is responsible for outputting the dense sequence of
TS-waypoints on the turbine blade (3-D freeform surface), according the HVOF
requirements. The workflow includes CAD sampling and normal vector estimation
with the OpenRave Open Dynamics Engine (ODE), samples pre-filtering, implicit
radial basis functions (RBF) for surface modelling \cite{c14} and surface
partition, and the intersection curve computation through the marching method
\cite{c15}.




\section{JOINT-SPACE TRAJECTORY GENERATION}\label{sec:ikfast}
The Joint-Space Trajectory Generation (JSTG) is a software component that
receive the TS-waypoints and outputs the JS-waypoints for the Joint-Space
Planner. Assuming that all TS-waypoints are in the manipulator range
(workspace), a TS-waypoint has, at least, one inverse kinematic solution for
a 6-DOF robotic arm. Position and orientation tolerances or redundant
manipulators provide infinite IK solutions for each TS-waypoint, thus random
sampling configurations may be computed for the planner.

The HVOF process admits an orientation tolerance of $\pm 30^\circ$
with respect to the normal vectors of the surface, and a position
(distance) tolerance of $\pm 20\,mm$ with respect to the surface.

\section{JOINT-SPACE PLANNER}\label{sec:dijkstra}

\section{MOVING LEAST SQUARE SMOOTH INTERPOLATOR}\label{sec:mls}


\section{SIMULATION AND RESULTS}\label{sec:results} 

\section{CONCLUSION AND FUTURE WORK}

\begin{thebibliography}{99}

\bibitem{c1} Freitas, Renan S., et al. "State of the Art and Conceptual Design
of Robotic Solutions for In Situ Hard Coating of Hydraulic Turbines." Journal of
Control, Automation and Electrical Systems 28.1 (2017): 105-113.


\bibitem{c2} Chen, Heping, et al. "Automated robot trajectory planning for spray
painting of free-form surfaces in automotive manufacturing." Robotics and
Automation, 2002. Proceedings. ICRA'02. IEEE International Conference on. Vol.
1. IEEE, 2002.

\bibitem{c3} Bo, Zhou, et al. "Fast and templatable path planning of spray
painting robots for regular surfaces." Control Conference (CCC), 2015 34th Chinese. IEEE, 2015.


\bibitem{c4} Tewolde, Girma S., and Weihua Sheng. "Robot path integration in
manufacturing processes: Genetic algorithm versus ant colony optimization." IEEE
Transactions on systems, man, and cybernetics-Part A: systems and humans 38.2
(2008): 278-287.

\bibitem{c5} Carr, Jonathan C., et al. "Reconstruction and representation of 3D
objects with radial basis functions." Proceedings of the 28th annual conference
on Computer graphics and interactive techniques. ACM, 2001.

\bibitem{c6} Bajaj, Chandrajit L., et al. "Tracing surface intersections."
Computer aided geometric design 5.4 (1988): 285-307.

\bibitem{c7} Burget, Felix, Maren Bennewitz, and Wolfram Burgard. "BI 2 RRT*: An
efficient sampling-based path planning framework for task-constrained mobile manipulation." Intelligent Robots and Systems (IROS), 2016 IEEE/RSJ International Conference on. IEEE, 2016.

\bibitem{c8} Stilman, Mike. "Global manipulation planning in robot joint space
with task constraints." IEEE Transactions on Robotics 26.3 (2010): 576-584.

\bibitem{c9} Oriolo, Giuseppe, and Christian Mongillo. "Motion planning for
mobile manipulators along given end-effector paths." Robotics and Automation, 2005. ICRA 2005. Proceedings of the 2005 IEEE International Conference on. IEEE, 2005.

\bibitem{c10} Ude, Ales, Christopher G. Atkeson, and Marcia Riley. "Planning of
joint trajectories for humanoid robots using B-spline wavelets." Robotics and Automation, 2000. Proceedings. ICRA'00. IEEE International Conference on. Vol. 3. IEEE, 2000.

\bibitem{c11} Levin, David. "The approximation power of moving least-squares." Mathematics of Computation of the American Mathematical Society 67.224 (1998): 1517-1531.

\bibitem{c12} Hauser, Kris, and Victor Ng-Thow-Hing. "Fast smoothing of
manipulator trajectories using optimal bounded-acceleration shortcuts." Robotics and Automation (ICRA), 2010 IEEE International Conference on. IEEE, 2010.

\bibitem{c13} Diankov, Rosen, and James Kuffner. "Openrave: A planning
architecture for autonomous robotics." Robotics Institute, Pittsburgh, PA, Tech.
Rep. CMU-RI-TR-08-34 79 (2008).

\bibitem{c14} Carr, Jonathan C., et al. "Reconstruction and representation of 3D
objects with radial basis functions." Proceedings of the 28th annual conference
on Computer graphics and interactive techniques. ACM, 2001.

\bibitem{c15} Bajaj, Chandrajit L., et al. "Tracing surface intersections."
Computer aided geometric design 5.4 (1988): 285-307.

\end{thebibliography}





\end{document}
